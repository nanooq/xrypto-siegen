% hitex:master
\documentclass[10pt]{beamer}
\usepackage[utf8]{inputenc}
\usepackage[ngerman]{babel}
\usepackage{wasysym}
\usepackage[locale=DE]{siunitx}
\newcommand{\bit}{bit}
\newcommand{\byte}{B}
\usepackage{tikz}
\usetikzlibrary{arrows}
\usetikzlibrary{positioning}
\usepackage{graphicx}

\usetheme{Luebeck}
\usecolortheme{dolphin}
\beamertemplatenavigationsymbolsempty
\setcounter{tocdepth}{2}
%\AtBeginSection[]{\frame{\tableofcontents[sectionstyle=show/shaded,subsectionstyle=show/show/hide]}}
\AtBeginSubsection[]{\frame{\tableofcontents[sectionstyle=show/shaded,subsectionstyle=show/shaded/hide]}}
%\beamerdefaultoverlayspecification{<+->}

\newcommand{\thetitle}{Thema der Präsentation}
\title[\thetitle]{\huge{\thetitle}}
\author{Xrypto Siegen}
\date{\scriptsize\today}

\begin{document}
    \frame{\titlepage}
    \frame{\tableofcontents}
    %\frame{\tableofcontents[sections=1-2]}
    %\frame{\tableofcontents[sections=3-5]}

    \section{Spickzettel}\label{sect:exampless}

\subsection{Folien}\label{sect:examples.slides}
\frame{
    Oftmals sind Titel für Folien überflüssig -- speziell beim gewählten Layout, da die ZuhörerInnenschaft ja oben nachlesen kann, worum es gerade geht.

    Genau deswegen hat diese hier auch keine eigene Überschrift.
}

\begin{frame}{Minimalismus macht sexy!}
    Hin und wieder jedoch ist ein catchy Slogan angebracht.
    Das sieht dann so aus wie hier dargeboten.
\end{frame}

\subsection{Listen}\label{sect:examples.lists}
\frame{
    Diese Folie zeigt eine einfache Aufzählung mit allem
    \begin{itemize}
        \item Pi,
        \item Pa und
        \item Po.
    \end{itemize}
}

\frame{
    Diese Folie veranschaulicht einen (seltsamen) Algorithmus: \\
    Bekannterweise lernen Kinder hierzulande
    \begin{enumerate}
        \item gehen und sprechen,
        \item still sitzen und ruhig sein.
    \end{enumerate}
}

\subsection{Layout}\label{sect:examples.layout}
\frame{
    \begin{block}{Achtung}
        Dieser Text ist sehr wichtig!
    \end{block}
}

\frame{
    \begin{columns}[T]
        \column{.5\textwidth}
        Dies ist ein Text aufgeteilt in zwei Spalten.
        Manchmal ist das ganz schick.
        \column{.5\textwidth}
        Deswegen kommt das auf den Spickzettel.
    \end{columns}
}

    \section{Abschnitt-Template}\label{sect:template}

\subsection{Unterkapitel 1}\label{sect:template.fstsub}
\frame{
    Text
}

\subsection{Unterkapitel 2}\label{sect:template.sndsub}
\frame{
    Text
}


    \section{}
    \frame{
        \begin{center}
        {\Large\bfseries Vielen Dank für die Aufmerksamkeit.} \\[5mm]
        Fragen?
        \end{center}
    }

    \frame{
        \begin{description}
            \item[Xrypto Siegen] ist ein Gruppe engagierter Kryptographie-Enthusiasten und Datenschutz-Aktivisten von der Universität Siegen und dem Hackspace Siegen mit der Vision / Mission, in regelmäßigen Intervallen Crypto-Parties an ebenjenen Orten zu veranstalten, um Menschen auf die Gefahren der vernetzten Welt aufmerksam zu machen und ihnen das Thema "`Datenschutz"' ans Herz zu legen.
            \item[Hackspace Siegen] ist ein gemeinnütziger Verein mit dem Ziel der Wissensvermittlung und Förderung im Bereich Technik/Informatik. Zu diesem Zweck organisiert der "`HaSi"' Vorträge, Workshops, Schnupperkurse für Kinder/Jugendliche und ähnliche Veranstaltungen.
            Darüber hinaus bietet das Vereinsheim eine Infrastruktur für gemeinschaftliches arbeiten an Projekten.
        \end{description}
    }
\end{document}
